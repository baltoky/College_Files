\documentclass[a4paper,12px] {article}

\begin{document}

\title{Assignment 1}
\author{Cesar A. Santiago}
\date{\today}
\maketitle

\pagenumbering {roman}
\tableofcontents
\newpage
\pagenumbering{arabic}

\section{GPA histogram.}
\subsection{Part 1}
Whithin the categories of 2.45-2.65 and 2.65-2.85 there are the most student
per category.
\subsection{Part 2}
Both ranges of grades: 2.45-2.65 and 2.65-2.85 both have 7 / 30 students. Therefore the categories put together make up 14 / 30 students.
\subsection{Part 3}
The portion of students that had less that a 2.65 GPA consited of 16 / 30 students.

\section{Formulas for Mean, Variance, and Standard Variation.}
\begin{itemize}
\item $ \overline{y} = \frac{1}{n} \sum_{i = 1} ^ n y_i $
\item $ s ^ 2 = \frac{1}{n - 1} \sum_{i = 1} ^ n ( y_i - \overline{y})^2 $
\item $ s = \sqrt{s^2} = \frac{1}{n - 1}\sqrt{ \sum_{i = 1} ^ n ( y_i - \overline{y})^2i} $
\end{itemize}

\section{From the 44 observations, take the Mean, Variance, and Standard Deviation.}
\subsection{Consider the following 44 observations:}
Set S = \{0.07, 5.7, 6.2, 7.1, 7.6, 7.7, 7.8, 7.8, 7.9, 8.2, 8.3, 8.4, 8.6, 8.7, 8.8,
 8.8, 8.8, 8.8, 8.8, 8.9, 8.9, 9.0, 9.1, 9.1, 9.2, 9.2, 9.3, 9.4, 9.5, 9.6, 9.6, 10.2,
 10.3, 10.3, 10.5, 10.5, 10.6, 10.7, 10.9, 11.3, 11.5, 11.8, 12.4, 12.7\} \newline
Using the formulas from section 2;
\begin{itemize}
\item  For the Mean we conclude the following: \newline $ \overline{y} = \frac{1}{n} \sum_{i = 1} ^ n y_i = \textbf{9.058} $
\item For the Variance we conclude the following:\newline $ s ^ 2 = \frac{1}{n - 1} \sum_{i = 1} ^ n ( y_i - \overline{y})^2 = \textbf{4.092} $
\item For the Standard Deviation we conclude the following:\newline $ s = \sqrt{s^2} = \frac{1}{n - 1}\sqrt{ \sum_{i = 1} ^ n ( y_i - \overline{y})^2i}= \textbf{2.023} $
\end{itemize}

\section {Resting breathing Mean and Standard Deviation.}
\subsection {9.7 to 14.3 breaths per minute.}
On an average normal distribution the breathing rates between 9.7 and 14.3 which are $[\overline y - s, \overline y + s]$ is equivalent to about \textbf{68\%}.
\subsection {7.4 to 16.6 breaths per minute.}
On an average normal distribution the breathing rates between 7.4 and 16.6 which are $[\overline y - 2s, \overline y + 2s]$ is equivalent to about \textbf{95\%}.
\subsection {9.7 to 16.6 breaths per minute.}
On an average normal distribution the breathing rates between 7.4 and 16.6 which are $[\overline y - s, \overline y + 2s]$ is equivalent to about \textbf{81.5\%}.
\subsection {Breaths per minute lower than 5.1 and higher than 18.9.}
On an average normal distribution the breathing rates lower than 5.1 and higher than 18.9 which are $5.1 < x > 18.9$ is approximately \textbf{0.27\%}.

\section{Time spent using the Internet.}
\subsection{Value exactly one Standard Deviation below}
The value one Standard Deviation below the mean is 0 users. Because we can not go negative in this projection.
\subsection{What portion of users pend lees than the first Standard Deviation point}
Given an approximatelly normal distribution of this data the percentage of people that spend an amount lower than one Standard Deviation below the mean is 16\%.
\subsection{Is time spent online approximately normally distributed? Why?}
This data is not normally distributed and we can deduce that because a user can not spend a negative amount of hours on the Internet per year. And a normal distribution would sugest that there
are users that are under that value. Therefore this data is not normally distributed.

\section{Weekly maintenance cost for a factory.}
\subsection{Average spendature: \$420 with a Standard Deviation of \$30}
The total amount that add when we add the mean and one standard deviation is \$450. The probability that next week will go over that budget is 16\%.
Since this data is normally distributed then the approximate percentage of datapoints above ($\overline y + s$) is our previously mentioned value of 16\%.

\section{Use of sumations to calculate variance $s^2$}
\subsection{Using the results given show that}
\label{sec7.1}
$$s^2 = \frac{1}{n-1}\sum_{i=1}^n(y_i-\overline y)^2 = \frac{1}{n-1}[\sum_{i=1}^n y^2 - \frac{1}{n}(\sum_{i=1}^n y_i)^2]$$
Work:\newline
$$ s^2 =  \frac{1}{n-1}\sum_{i=1}^n(y_i-\overline y)^2 = \frac{1}{n-1}[\sum_{i=1}^n (y_i)^2 - \sum_{i=1}^n (\overline y)^2]$$ In this first line we distribute the parenthesis to allow more algebraic mobility.
$$ => \frac{1}{n-1}[\sum_{i=1}^n (y_i)^2 - \sum_{i=1}^n (\frac{1}{n} \sum_{i=1}^n (y_i)^2)]$$ We get this result by using the Mean function from Section 2.
$$ => \frac{1}{n-1} [\sum_{i=1}^n y_i ^2 - \frac{1}{n} ( \sum_{i=1}^n y_i)^2]$$ This final result uses the second function provided to reach the function that we desired.
\subsection{Using the result from \ref{sec7.1} find the Standard Deviation with the following values:}
Let S be the set of sample measurements: S = \{1, 4, 2, 1, 3, 3\}\newline
From the previous section we can surmise that:
$$s^2 = \frac{1}{n-1} [\sum_{i=1}^n y_i ^2 - \frac{1}{n} ( \sum_{i=1}^n y_i)^2]$$
When we plug in our set of values S:
$$s^2 = \frac{1}{6-1}[(1^2 + 4^2 + 2^2 + 1^2 + 3^2 + 3^2) - \frac{1}{6} (1 + 4 + 2 + 1 + 3 + 3) ^ 2]$$
$$ => \frac{1}{5} (40 - \frac{14^2}{6})$$
$$ => \frac{1}{5} (7.33)$$
$$ => \textbf{1.46}$$

\section{Family that contains two children, an older child and a younger one.}
Let S be the set of potential child combinations
$$S = \{FF, MM, FM, MF\}$$ \newline
Let A be the set of elements with no males; $$ A = \{FF\}$$
Let B be the set of elements with two males; $$ B = \{MM\}$$
Let C be the set of elements with at least one male; $$ C = \{MM, FM, MF\}$$\newline
The following are sets that result from the combinations of the previous.
$$ A \cap B = \{ \} = \emptyset$$
$$ A \cup B = \{ FF MM \}$$
$$ A \cap C= \{ \} = \emptyset$$
$$ A \cup C = \{ FF, MM, FM, MF \}$$
$$ B \cap C = \{ MM \}$$
$$ B \cup C = \{ MM, FM, MF \}$$
$$ C \cup \overline B = \{ FM, MF \}$$


\end{document}