\documentclass[a4paper, 12pt]{article}
\begin{document}

\title{Assignement 2}
\author{Cesar A. Santiago}
\date{\today}
\maketitle

\pagenumbering{roman}
\tableofcontents
\newpage
\pagenumbering{arabic}

\section{Advertising agency.}
Given by the problem:\newline
P(seeing the ad in a Magazine) = $\frac{1}{50}$\newline
P(seeing ad on T.V.) = $\frac{1}{5}$\newline
P(seeing the ad on both) = $\frac{1}{100}$\newline
P(the customer buys / sees the ad) = $\frac{1}{3}$\newline
P(the customer buys / not seen the ad) = $\frac{1}{10}$\newline
What is the probability that a randomly selected customer will buy the product?\newline
P(customer buys the product) = P(customer buys / sees the ad) + P(customer buys / not seen the ad) = $\frac{13}{30}$\newline
The reasoning is because the set that have seen the ad and those who have not are mutually exclusive.
\section{Median income for U.S. families.}
\subsection{Points in the sample space.}
Let O be the event of an income over the median, and E be the event of an income equal to or under the median.\newline
Let S be the sample set of the events that can occur if we pick four families within the census: \{OOOO, OOOE, OOEO, OOEE, OEOO, OEOE, OEEO, OEEE, EOOO, EOOE, EOEO, EOEE, EEOO, EEOE, EEEO, EEEE\}\newline
$\|S\| = 16$
\subsection{Simple events in each of the following:}
A) At least two incomes exceeding \$43,318 : $$\|\{OOOO, OOOE, OOEO, OEOO, OEEO, EOOO, EOOE, EOEO, EEOO\}\| = 9$$
B) Exactly two had incomes exceeding \$43,318: $$\|\{OEEO, EOOE, EOEO, EEOO\}\| = 4$$
C) Exaclt one had income less that or equal to \$43,318: $$\|\{OEEE, EOEE, EEOE, EEEO\}\| = 4$$
\subsection{Probabilities of the events}
The probabilities of these events happening are the sum of probabilities of simple events times each of the events probabilities.
In this case that ends up being the sum of simple events over the total because they have similar probability.\newline
P(A): $ \frac{9}{16}$, 
P(B): $\frac{4}{16}$, 
P(C): $\frac{4}{16}$
\section{Imported wines.}
Let A, B, C  be different brands of wine. The set S is the set of wines ranked from best to worst = $\{ABC, ACB, BAC, BCA, CAB, CBA\}$ with a size of 6.
Assuming equal probability of being picked, Let A be the markably best wine. The probability that A will be picked last is the set \{BCA, CBA\} which has a size of 2.
The probability of any element in this set being picked is $P(A) = \frac{2}{6}$.
\section{Demand for a tool.}
The following are the mean and variance of the times the tool is used. We multiply these values by \$10 to find the dollar amount.
$$E(Y) = \sum_{all y} yp(y) = \sum_{y = 0}^2 yp(y) = 0*0.1 + 1*0.5 + 2*0.4 = 1.3 = \mu$$
$$Var(Y) = E(Y^2) - \mu^2 = \sum_{y=0}^2 y^2p(y) - 1.3^2 = (0^2*0.1 + 1^2*0.5+2^2*0.4) - 1.3 = 2.1 - 1.3 = 0.8$$
$$E(Y) = \$13, V(Y) = \$8 $$
\section{Multiple choice examination.}
Let Y be the random variable Y: The number of questions that are answered correctly.
Then we can determine that assuming true randomness then $P(Y\leq1) = p(1) = \frac{1}{5}$
Then we try to find $P(Y\geq10) = 1 - P(Y<10) = 1 - (P(Y = 10) + P(Y = 9) + P(Y = 8) + P(Y = 7) + P(Y = 6) + P(Y = 5) + P(Y = 4) + P(Y =3) + P(Y = 2) + P(Y = 1))
 = 1 - (0.0001 + 0.0006 + 0.0034 + 0.0138 + 0.0429 + 0.1031 + 0.187 + 0.25 + 0.23 + 0.13) = 1 - 0.96 = \textbf{0.0391}$
\section{New surgical procedure.}
Let Y denote a random variable that follows this distribution; $Y \sim Binomial(n, p)$.\newline
$P(Y = y) = (^n_y)p^y(1-p)^{n-y}$\newline
The probability that all five are successful if p = 0.8:$$ P(Y = 5) = (^5_y)0.8^y(1-0.8)^{5-y} = 0.327$$
\newline The probability that exactly four are successful if p=0.6: $$P(Y = 4) = (^5_y)0.6^y(1-0.6)^{5-y} = 0.2592$$
\newline The probability that less than two are successful if p = 0.3: $$P(Y<2) = P(Y \leq 1) = P(Y = 1) + P(Y = 0) = 0.36015 + 0.16807 = 0.52822$$
\section{Applicants for an industrial job.}
Let $Y \sim Geometric(p)$ be our Random Variable in this distribution with a p of $\frac{3}{10}$. Where Y is the number of the first applicant interviewed that has advanced training in computer science, begining at one and counting up.\newline\newline
\indent The probability that the first applicant with advanced programming training is found in the fifth interview: $$P(Y = 5) = (1 - p)^{y-1}p = (\frac{7}{10})^4(\frac{3}{10}) = 0.07203$$
\indent Find the Expected Value of this distribution: $$E(Y) = \frac{1}{p} = \frac{1}{\frac{3}{10}} = 3.3\overline3$$
\section{Oil prospector.}
Let $Y \sim Geometric(p)$ be our Random Variable for this distribution where p = 0.2.\newline
\newline The probability that the third hole drilled is the first in the sequence to yield a productive well is: $$P(Y = 3) = (1-p)^{y-1}p = 0.8^3*0.2 = 0.1024$$
What is then, the probability that the prospector will fail to find a well if at most ten trials can be afforded? 
$$P(Y > 10) = 1 - P(Y \leq 10) = $$ $$=> 1 - (p(10) + p(9)+p(8)+p(7)+p(6)+p(5)+p(4)+p(3)+p(2)+p(1)))$$
$$=> 1 - ((1-0.2)^{9}0.2 + (1-0.2)^{8} 0.2 + (1-0.2)^{7}0.2 + (1-0.2)^{6}0.2 + (1-0.2)^{5}0.2$$$$ + (1-0.2)^{4}0.2 + (1-0.2)^{3}0.2 + (1-0.2)^{2}0.2 + (1-0.2)0.2 + )
(1-0.2)^{0}0.2$$
$$=> 1 - (0.0268 + 0.0335 + 0.04194 + 0.0524 + 0.0655 + 0.0819 + 0.1024 + 0.128 + 0.16 + 0.2) = 0.107$$
\section{Defective printing machines.}
The probability of a defective machine given is $\frac{2}{5} = 0.4$\newline
Let Y be our Random Variable as Y: The number of machines selected that are not defective. Y will simulate a Binomial distribution: $Y \sim Binomial(n, p)$ for n = 5 and p = 0.6.\newline
$$P(Y = y) = (^n_y)p^y(1-p)^{n-y}$$
Therefore the probability that all machines are functioning is denoted by:
$$P(Y = 5) = (^5_5)*0.6^5*0.4^0 = 0.07776$$
\section{Poisson distribution.}
Let Y denote a random variable that has a Poisson distribution with mean $\lambda = 2$
$$P(Y = y) = \frac{1}{y!}\lambda^ye^{-\lambda}$$
Then the following probabilities are as such:
$$P(Y = 4) = \frac{1}{4!} * 2^4 * e^{-2} = 0.0.090$$\newline
$$P(Y \geq 4) = 1 - P(Y < 4) = 1 - (P(Y = 3) + P(Y = 2) + P(Y = 1))$$ $$ = 1- (0.180 + 0.270 + 0.270) = 1 - 0.72 = 0.28$$\newline
$$P(Y < 4) = (P(Y = 3) + P(Y = 2) + P(Y = 1))$$ $$ = (0.180 + 0.270 + 0.270) = 0.72 $$\newline
$$V(Y) = E(Y^2) - \mu^2 = E(Y^2) - \lambda^2$$
To find the variance we must first find $E(Y^2-Y)$
The reason for this is that we can then say $E(Y^2) = E(Y^2-Y) + E(Y)$
So Then $E(Y^2-Y) = \sum_{y=1}^n y(y-1) \frac{1}{(y)(y-1)(y-2)!} * \lambda^{y-2}* \lambda^2 * e^{-lambda}$
After some cancelation we arrive to the output of $=> \lambda^2 \sum_{y=3}^n\lambda^{y-2}e^{-\lambda}$
So from this we know $E(Y^2-Y) = \lambda^2$
And $E(Y^2) = \lambda^3$
Therefore The variance is: 
 $$E(Y^2) - \lambda^2 = \lambda^3 - \lambda^2 = \lambda$$
\section{Customers at a checkout counter.}
Let $Y \sim Poisson(\lambda)$ Given that $\lambda = 4$ what are the following probabilities?
$$P(Y \leq 9) = $$ $ P(Y = 9) + P(Y = 8) + P(Y = 7) + P(Y = 6) + P(Y = 5) + P(Y = 4) + P(Y = 3) + P(Y = 2) + P(Y = 1)  = 0.013 + 0.029 + 0.059 + 
0.104 + 0.156 + 0.195 + 0.195 + 0.1465 + 0.0732 = \textbf{0.9707}$
$$P(Y \geq 5) = $$ $1 - (P(Y = 5) + P(Y = 4) + P(Y = 3) + P(Y = 2) + P(Y = 1)) = 1 - (0.156 + 0.195 + 0.195 + 0.1465 + 0.0732) = 1 - 0.7657 = \textbf{0.2343}$
$$P(Y = 9) = \frac{1}{9!} * \lambda^9 * e^{-\lambda} = \textbf{0.013}$$
\section{Consider Y as a binomial random variable.}
$Y \sim Binom(n, p)$. We know that E(Y ) = np and V (Y ) = npq. We will use mgf(t) to prove them again.
If $Y \sim Binomial(n, p)$ the we know that $P(Y= y) = (^n_y)p^y(1-p)^{n-y}$\newline
We can obtain the mgf as: $$ mgf_Y(t) = E(e^{ty}) = \sum_{y = 0}^n(^n_y)e^{ty}p^y(1-p)^n-y$$
$$ = \sum_{y=0}^n (^n_y)(e^tp)^y(1-p)^{n-y}$$
Here we can use the Binomial Theorem $\sum_{y=0}^n(^n_y)x^yz^{n-y} = (x + z)^n$
and we conclude that $mgf_Y(t) = (e^tp + (1-p))^n = (q +pe^t)^n$
\newline To fim the Expeted Value and Variance we will use derivatives:\newline
$E(Y) = \frac{d}{dt} mgf_Y(t) = \frac{d}{dt} (1 +  \frac{tE(Y)}{})= n(pe^t) = np$\newline
$V(Y) = \frac{d^2}{d^2t} mgf_Y(t) = \frac{d^2}{d^2t} (q + pe^t)^n = npq$

\end{document}

















